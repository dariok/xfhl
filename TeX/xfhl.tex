\documentclass[open=any, index=totoc, paper=167mm:239mm]{scrbook}
\KOMAoption{headings}{small,optiontohead}

% Set fonts
\usepackage{fontspec}
\setmainfont[UprightFeatures={
    SmallCapsFont={FreeSerif},
    SmallCapsFeatures={Letters=SmallCaps,LetterSpace=6}
  },
  BoldFeatures={
    LetterSpace=3
  }
]{FreeSerif}
\setsansfont{FreeSans}
\setmonofont{FreeMono}

% Set main page layout
\usepackage[xetex]{geometry}
\geometry{inner=20mm, outer=35mm, top=22.5mm, bottom=30mm,
  headsep=2.5mm, marginparsep=3mm, marginparwidth=30mm}%, showframe}

% set headers
\usepackage{scrlayer-scrpage}
\automark[chapter]{chapter}
\automark*[section]{}

% Structure: settings for page style, font and toc entry
% Part
\renewcommand*{\partpagestyle}{empty}

% Chapter: starts at the top of the page
\RedeclareSectionCommand[%
  afterskip=0.5\baselineskip plus .15\baselineskip minus .15\baselineskip,
  beforeskip=0\baselineskip, %1\baselineskip
  innerskip=0pt,
  tocnumwidth=4em
]{chapter}

% thumbmarks
  \usepackage{tikz}

  % taken from https://tex.stackexchange.com/questions/541881
  \newcommand*{\firstpartthumbskip}{.1\paperheight}
  \newcommand*{\lastpartthumbskip}{\firstpartthumbskip}
  \newcommand*{\partthumbheight}{.1\paperheight}
  \newcommand*{\partthumbwidth}{.01\paperheight}
  \newcommand*{\partthumbskip}{.1\paperheight}
  \colorlet{partthumbboxcolor}{gray!30}
  \newcommand*{\partthumbcolor}{white}
  \newcommand*{\partthumbformat}{\thepart}
  \newkomafont{partthumb}{\normalfont\Large\color{\partthumbcolor}}

  \makeatletter
  \newcommand*\partthumb@box{%
    \usekomafont{partthumb}%
    \parbox[c][\partthumbheight][c]{\partthumbwidth}{%
      \centering%
      \begin{tikzpicture}%
        \node[circle, minimum width=2.5cm, minimum height=2.5cm, fill=partthumbboxcolor]
        {\ifodd\value{page}\makebox[0pt][c]{\hspace{-1cm}\partthumbformat} \else\makebox[0pt][c]{\hspace{1cm}\partthumbformat}\fi};
      \end{tikzpicture}%
    }%
  }
  \newcommand*{\partthumbbox}{%
    \if@mainmatter
      \ifnum\value{part}>\z@
        \ifnum \value{partthumb}<\z@
        \else
          \begingroup
            \protected@edef\reserved@a{\partthumbformat}%
            \ifx\reserved@a\lastpartthumbformat\else
              \stepcounter{partthumb}%
              \global\let\lastpartthumbformat\reserved@a%
            \fi
            \@tempcnta=\numexpr
              \dimexpr
                \paperheight
                -\firstpartthumbskip
                -\partthumbwidth
                -\lastpartthumbskip
              \relax / \dimexpr
                \partthumbskip
              \relax
                +1
              \relax
            \ifnum \value{partthumb}<\@tempcnta
            \else
              \setcounter{partthumb}{0}%
            \fi
            \vspace*{%
              \dimexpr
              \firstpartthumbskip
              + ( \partthumbskip )
              * \value{partthumb}%
              - \baselineskip
              \relax
            }\par
            \setlength{\fboxsep}{0pt}%
            \ifodd\value{page}
              \hfill
              \makebox[0pt][r]{%
                \rotatebox[origin=c]{0}{\partthumb@box}}%
              \hspace*{0.9cm}
            \else
              \hspace*{-1.4cm}
              \makebox[0pt][l]{%
                \rotatebox[origin=c]{0}{\partthumb@box}}%
            \fi
          \endgroup
        \fi
      \fi
    \fi
  }
  \makeatother

  \newcounter{partthumb}
  \setcounter{partthumb}{10000}
  \newcommand*{\lastpartthumbformat}{\relax}

  \DeclareNewLayer[%
    background,%
    outermargin,%
    contents=\partthumbbox
  ]{partthumb}

  \newcommand*\EnablePartthumb{%
    \IfLayerAtPageStyle{scrheadings}{partthumb}{}
    {\AddLayersToPageStyle{@everystyle@}{partthumb}}%
  }
  \newcommand*\DisablePartthumb{%
    \RemoveLayersFromPageStyle{@everystyle@}{partthumb}%
  }
  
\EnablePartthumb
% END thumbmarks

% restart chapter counting for every part
\usepackage{chngcntr}
\counterwithin*{chapter}{part}

% language setting
\usepackage{polyglossia}
\setmainlanguage[variant=british]{english}

% define indices
\usepackage[split]{splitidx}                                                                                             
\makeindex                                                                                                               
\newindex[Topics]{top}

% additional packages
\usepackage{hyperref}
\usepackage{microtype}
\usepackage{graphicx}

\begin{document}
  \frontmatter  
    \thispagestyle{empty}
    \vspace*{\fill}
    \begin{center}The XML Solar System for Humanists and Librarians\end{center}
    \vspace*{\fill}

    \tableofcontents

  \mainmatter
    \part{XML}
    \chapter{Basics}

\section{What is XML?}
\par The Extensible Markup Language, or XML, is a formalised language, originally
               defined by the World Wide Web Consortium (W3C) in 1998. It is by far not the
               oldest markup language but one of the most frequently used markup languages today. When it was designed,
               several design principles were closely followed to improve other, older markup languages. This careful
               design process is one of the reasons of its success. To better understand the design principles, we need
               to take a closer look at the basic reasons for using and concepts behind markup languages in general.
\subsection{Markup languages}

\section{Introducing the DOM}

\section{Serialisations}

    \part{XPath}

    \part{XSLT 1}
    %\chapter{Basics}

\section{What is XML?}
\par The Extensible Markup Language, or XML, is a formalised language, originally
               defined by the World Wide Web Consortium (W3C) in 1998. It is by far not the
               oldest markup language but one of the most frequently used markup languages today. When it was designed,
               several design principles were closely followed to improve other, older markup languages. This careful
               design process is one of the reasons of its success. To better understand the design principles, we need
               to take a closer look at the basic reasons for using and concepts behind markup languages in general.
\subsection{Markup languages}

\section{Introducing the DOM}

\section{Serialisations}

    \part{XPath functions 1}

    \part{XSLT 2}

    \part{XPath functions 2}

    \part{Unit testing for XSLT}

    \part{XQuery}

    \part{TEI}

    \part{Indices}
    \printindex*
\end{document}